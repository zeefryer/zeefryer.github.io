%\documentclass[10pt,a4paper]{article}
\documentclass[11pt]{amsart}

\usepackage{amssymb}
\usepackage{amsmath}
\usepackage{amsthm}
\usepackage[pdftex]{graphicx} 
\usepackage{tikz}
\usepackage{enumerate}
\usepackage{url}


\tolerance = 10000

\setlength{\oddsidemargin}{8mm}
\setlength{\evensidemargin}{8mm}
\setlength{\topmargin}{-5mm}

\setlength{\textwidth}{145mm}
\setlength{\textheight}{220mm}


\linespread{1.1}

\setlength{\parindent}{0pt}
\setlength{\parskip}{1.5ex plus 0.5ex minus 0.2ex}

\newtheorem{theorem}{Theorem}[section]
\newtheorem{lemma}[theorem]{Lemma}
\newtheorem{corollary}[theorem]{Corollary}
\newtheorem{proposition}[theorem]{Proposition}
\newtheorem{conjecture}[theorem]{Conjecture}
\theoremstyle{definition}
\newtheorem{example}[theorem]{Example}
\newtheorem{remark}[theorem]{Remark}
\newtheorem{definition}[theorem]{Definition}
\newtheorem{question}[theorem]{Question}
\newtheorem{questions}[theorem]{Questions}
\newtheorem{problem}{Problem}

\numberwithin{equation}{theorem}


\begin{document}
\title{$H$-primes and their applications: from algebra to quantum field theory}
\author{Si\^an Fryer}
\maketitle

\vspace{-2em}
\section{Lay summary}

Given a $2\times 2$ matrix, can you find four nonnegative numbers to fill it so that the determinant is also nonnegative? What about an $n \times n$ matrix, such that whichever subset of rows and columns are chosen, the determinant of the resulting matrix is still nonnegative?  Matrices of this form are called totally nonnegative (TNN) and they show up throughout mathematics and physics.  Interesting questions include:

\begin{itemize}
\item Given a matrix, how can we check if it is TNN without computing the determinants of every possible submatrix (or \textit{minor}) of the matrix?
\item Given a list of minors, can we construct a matrix in which these are zero and all other minors are positive?
\end{itemize}

About five years ago, the study of TNN matrices was shown to be connected to a completely different area of mathematics: noncommutative algebra.  This is the study of algebraic structures where the order of multiplication matters: the product $a*b$ need not always be the equal to the product $b*a$. For certain nice noncommutative algebras known as quantum groups, we can learn a lot about the algebra by studying its $H$-primes: objects as fundamental to these algebras as prime numbers are to the integers.  

These $H$-primes turn out to share a lot of similar behaviour with total nonnegativity, which means we can start using techniques developed for one in order to study the other.  My project involves studying the behaviour and structure of these quantum groups via their $H$-primes, but I intend to solve it by finding new ways to move between the two languages and then combining the strengths of both.

Finally, there are close links from these areas to physics: for example, diagrams for studying the interaction of particles in a simplified model of quantum field theory can be elegantly rephrased in terms of total nonnegativity.  TNN matrices also provide an excellent framework for modelling the interaction of shallow water waves, and understanding these interactions is a crucial part of tsunami prediction.  I intend to use this physical intuition to inform my study of $H$-primes, and then feed these algebraic results back into the physics to obtain new results in this area as well.

\section{Research Plan}
%\pagenumbering{gobble}
\subsection{Background} In the study of noncommutative algebra, a key problem is to understand the \textit{prime ideals}: certain subsets of the algebra which generalise the role that prime numbers play in the integers.  

Of particular interest to both mathematicians and physicists is the family of noncommutative algebras known as \textit{quantum groups}, which are inspired by well known (semi-)groups such as $M_{m,n}$ (all $m\times n$ matrices) and $SL_n$ ($n \times n$ matrices with determinant 1).  Understanding the prime ideals of these algebras will give us a much clearer insight into their properties and structure.  

The behaviour of the prime ideals in a quantum group is governed by finitely many ``special'' ones called $H$-primes.  These $H$-primes have been intensively studied in recent years, and the collections of primes associated to each individual $H$-prime (known as the \textit{strata}) are now reasonably well understood.  What is still lacking, however, is an understanding of how these strata interact with each other: intuitively, we have a finite number of building blocks and an idea of what the final shape should be, but we don't yet know how they actually fit together.

%This question has been studied by a number of experts in the field, including Brown, Cauchon, Goodearl, Hodges, Joseph, Letzter, Levasseur and Yakimov (see e.g. \cite{GoodearlSummary,yakimov} and references therein), but so far remains unsolved except in a few simple cases.

\begin{problem}\label{problem quantum structure}
For any quantum group $A$, describe the topological structure of its set of prime ideals.
\end{problem}

A common feature of quantum groups is a ``deformation parameter'' $q$, which controls the noncommutativity of the algebra.  By letting $q$ tend to 1 we obtain a commutative Poisson algebra with some of the same structure as the original algebra, but the underlying commutativity means that it is amenable to study by a different set of techniques. 

\begin{problem}\label{problem homeomorphism}
Let $A$ be a quantum group, and $R$ the algebra obtained by letting $q$ tend to 1. Find a homeomorphism (a structure-preserving map) between the set of prime ideals in $A$ and the corresponding set in $R$.
\end{problem}

These problems are some of the major motivating questions in the study of the noncommutative geometry of quantum groups today.  In \cite{Me2}, I answered Problems~\ref{problem quantum structure} and \ref{problem homeomorphism} for the quantum group corresponding to $SL_3$: this was the first progress on this question for algebras of quantum matrices since the case of $SL_2$ was described by Hodges and Levasseur in 1993.

Building on this success, I intend to first solve Problems~\ref{problem quantum structure} and \ref{problem homeomorphism} for all algebras of quantum matrices, and then extend these results to cover all quantum groups.  To achieve this, I will make use of recent exciting developments connecting noncommutative algebra to several other areas of mathematics and physics.  A brief introduction to these topics is given in \S\ref{s:physics}, and \S\ref{ss:my work} outlines my existing and planned future work.


\subsection{Total Nonnegativity and Applications to Physics}\label{s:physics} In 2011, Goodearl, Launois, and Lenagan showed that there is an extremely close link between $H$-primes in quantum matrices and cells of totally nonnegative (TNN) real matrices, i.e. matrices in which every minor is positive or zero \cite{GLL}.  This allows us to use combinatorial techniques to study questions in quantum groups and vice versa, to the benefit of both areas. 

The combinatorics of TNN matrices (and more generally, total nonnegativity in the real Grassmannian $Gr(k,n)$) have also been recently linked to several areas of physics. For example, they can be used to model the behaviour of shallow water waves, which has important implications for understanding the behaviour of tsunamis. Another application appears in the $N=4$ Supersymmetric Yang Mills ($N=4$ SYM) model of quantum field theory, where certain TNN cells in $Gr(k,n)$ correspond to Wilson loop diagrams \cite{agarwala}.  These are a dualisation of Feynman diagrams in $N=4$ SYM which are used to compute interactions of particles.

\begin{problem}\label{problem physics}
Describe the geometric structure of admissible Wilson loop diagrams and their boundaries in $Gr(k,n)$.
\end{problem}  

$N=4$ SYM is of great interest in physics at the moment: it is a simplified model of quantum field theory used as a ``testing ground'' for new theories.  Identifying which Wilson loop diagrams share geometric boundaries, and hence cancel each other out in later computations, will provide the tools required to efficiently compute scattering amplitudes of certain particle interactions in $N=4$ SYM.  

By translating Problem~\ref{problem physics} into the language of $H$-primes and total nonnegativity, we gain a host of new techniques which will allow us to describe the behaviour of admissible Wilson loop diagrams.

\subsection{New Approaches combining Algebra, Combinatorics, and Physics} \label{ss:my work} In \cite{Me2}, I showed that recent results of Brown and Goodearl in \cite{GBrown} provide a framework which allows us to solve Problems \ref{problem quantum structure} and \ref{problem homeomorphism}; this was achieved by applying the techniques directly to a specific example of a quantum group.  The next step is to generalise these ideas to more general quantum groups, a program which I have already begun.  In joint work with Casteels, we used a combination of algebraic and combinatorial tools in order to generalise the first key step (the construction of denominator sets) to all algebras of quantum matrices \cite{Me3}. In collaboration with Yakimov, we generalised this further to cover a much wider class of algebras, including all quantum groups corresponding to simple Lie groups \cite{MeMilen}.

With these results in hand, I plan to analyse the structure of the algebras $Z_{JK}$ defined in \cite{GBrown}, which hold the key to understanding the interactions between the different strata of prime ideals.  This would allow us to give a full answer to Problems~\ref{problem quantum structure} and \ref{problem homeomorphism} for all quantum groups.

Our work in \cite{Me3,MeMilen} greatly simplifies the description of these algebras compared to the original definition given by Brown and Goodearl, allowing us to apply a combination of algebraic and combinatorial techniques to the problem in completely new ways.  By bringing the TNN viewpoint into the picture, I also expect to be able to apply the methods and ideas of cluster algebras to this problem.


Finally, Problem~\ref{problem physics} should be especially amenable to algebraic and combinatorial techniques, since the subset of these diagrams which are the hardest to study in purely physical terms turn out to be the most well-understood from a mathematical point of view!  In forthcoming work with Agarwala, we develop the basic combinatorial tools required to study Wilson loop diagrams in terms of total nonnegativity; these respresent a significant improvement on current techniques, which are computational and therefore restricted to low-dimensional examples only.

These interactions between algebra, combinatorics, and physics are extremely new and exciting, and we expect the combination of all three to be very successful in tackling each of the problems outlined above.  





%\bibliographystyle{plain}
%\bibliography{bibliography} 

\begin{thebibliography}{1}

\bibitem{agarwala}
S. Agarwala, E. Marin Amat.
\newblock Wilson loop diagrams and positroids.
\newblock {\em Commun. Math. Phys.} 2016.

\bibitem{GBrown}
K. A. Brown, K. R. Goodearl.
\newblock Zariski topologies on stratified spectra of quantum algebras.
\newblock \textit{MSRI Publ.} Vol 68.
\bibitem{Me3}
K. Casteels, S. Fryer.
\newblock From Grassmann necklaces to restricted permutations and back again,
\newblock to appear in {\em Algebras and Representation Theory.}

\bibitem{Me2}
S. Fryer.
\newblock The prime spectrum of quantum {$SL_3$} and the {P}oisson-prime
  spectrum of its semi-classical limit,
\newblock to appear in {\em  Trans. Lond. Math. Soc.}

\bibitem{MeMilen}
S. Fryer, M. Yakimov.
\newblock Separating Ore sets for prime ideals of quantum algebras,
\newblock  to appear in {\em Bull. Lond. Math. Soc.}

%\bibitem{GoodearlSummary}
%K.~R. Goodearl.
%\newblock Semiclassical limits of quantized coordinate rings.
%\newblock In {\em Advances in ring theory}, Trends Math., 2010.

\bibitem{GLL}
K.~R. Goodearl, S. Launois, T. H. Lenagan.
\newblock Torus-invariant prime ideals in quantum matrices, totally nonnegative cells and symplectic leaves.
\newblock \textit{Math. Z.} 269, 2011.

%\bibitem{yakimov}
%M.~Yakimov.
%\newblock On the spectra of quantum groups
%\newblock {\em Mem. Amer. Math. Soc.} 229 (1078) 2014.

\end{thebibliography}

\end{document}